Aside from friends being able to look at one another's popularity, our analysis offers assistance in several different fields. Most prominently, advertising companies could utilize the ranking to find the most influential Twitter users and work with them to reach as many people as possible. Since every interaction on Twitter is available to everyone with their development system, anyone could perform similar analysis to ours.

As a company, Twitter could even find this process useful. Being able to find their most active users would be an incredibly useful tool for analysis of the functionality of the website. While we only sampled a very small subset of users, Twitter could include and compute the ranking with every user in the data set--something even more valuable.

Beyond these uses, an interesting abstract idea is the thought that due to applications of matrix methods, one can literally assign a human a ranking of influence. While not everyone in the world uses Twitter and our definition of influence is not absolute, it is still peculiar that an entire population can be ranked by popularity with ease, despite a large data set.

A simplistic demonstration of this potential service is available at \url{www.will-farmer.com/twilight}. This website was created by William Farmer, and uses a simplified algorithm to rank the users.
