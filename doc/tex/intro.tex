Twitter is a free, online social media website with over 550 million users worldwide. It allows a user to create brief messages, or ``tweets'' (not exceeding 140 characters in length), and share them with their followers. The popularity of this website has greatly altered the way in which information is transferred across the world, as well as the speed at which this is done. With that being said, any one user has the opportunity to become rather influential based on how many other users on Twitter interact with his/her tweets. Because of its clear relevance in the modern world, celebrities, bloggers, and news media sites have taken to sharing information with Twitter. Some especially famous celebrities have several millions of followers, and are obviously incredibly influential in the world of Twitter.

    We then seek to rank Twitter users based on how influential they are to other users. In order to do so, one must account for the three most common forms of interaction on the website: followers, favorites, and retweets.

        To understand these three interactions, take for example user A and user B. If user A ``follows'' user B on Twitter, anything that user A tweets will appear on user B's news feed. This is by far the most common form of interaction. If user A ``favorites'' a tweet from user B, user A expresses his/her interest in user B's tweet because of its humor, importance, relevance, etc. Users can see one another's favorited tweets, so if user B is commonly favorited there is a good chance others will see his/her tweets. Finally, the least common interaction, a ``retweet'' occurs when user B chooses to allow a tweet from user A to show on his/her page. In doing so, all of the followers of user B see the tweet that user A posted. This broadens the scope of their influence, as many more followers are able to see the tweet without having to search for it. This can quickly allow a tweet from someone with a small amount of followers to reach a large amount of users.

                Due to the immense amount of users in Twitter, an analysis of the entire population would be very time consuming and difficult. We therefore limit the scope of our analysis to a random sample of tweets in the Boulder area. However, since these tweets could be retweets of users outside of Boulder, or even the country, our data set will almost certainly include a broad spectrum of users.
