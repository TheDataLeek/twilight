After implementing the power method and isolating the final dominant eigenvector, we had an explicit ranking of the users in our data set. An interesting note in this eigenvector was the large amount of users with negligible influence; out of nearly 3000 users, only 26 had an influence value greater than zero. This did not come as too much of a surprise due to the nature of our sample data set. Many of these users likely retweeted one of the more influential people and had no other interactions, diminishing their relative influence.

This clear difference in number of users with appreciable influence and those with almost no influence imply that the vast majority of users on Twitter are not all that influential. Our data suggests that in fact there are very few users who truly hold significant weight. As a matter of fact, we found one of the influential users in our data set was none other than Justin Bieber. While he does not live in Boulder, several people must have retweeted him and thereby introduced him to our data set. With his some 47 million followers, it does not come as a surprise that he was one of the most influential users on Twitter in our analysis.

This does relay a rather dismal message to almost the entire remainder of Twitter users however--they are just not that important. With the massive amount of followers and retweets that celebrities like Justin Bieber have, it is highly unlikely that a random user's tweets will reach the majority of the users on Twitter. However, this will certainly not keep people from using Twitter. Many users have an account simply to follow celebrities or comedians. Other have one to keep up with their sports teams or local news stations. Having a Twitter account is not always about trying to be the most popular user. Instead, it serves as a means of quick and easy communication for friends and celebrities alike.
